%!TEX root = ../template.tex
%%%%%%%%%%%%%%%%%%%%%%%%%%%%%%%%%%%%%%%%%%%%%%%%%%%%%%%%%%%%%%%%%%%%
%% Fz_help.tex
%% NOVA thesis document file
%%
%% Chapter with lots of dummy text
%%%%%%%%%%%%%%%%%%%%%%%%%%%%%%%%%%%%%%%%%%%%%%%%%%%%%%%%%%%%%%%%%%%%

\typeout{NT FILE Fz_help.tex}%


\chapter{Help}
\label{cha:help}


\section{===== Ordem dos Ficheiros =====}

\begin{verbatim}
    Letra   Nome                    Descrição
    A.0     Dedicatory              Dedicatória
    B.0     Acknowledgements        Agradecimentos
    C.0     Quote                   Citação
    D.x     Abstract                Resumo
        .1      pt
        .2      en
        .3      de
        .4      es
        .5      fr
        .6      gr
        .7      it
    E.x     Glossaries              Glossários
        .1      glossary        Glossário
        .2      acronyms        Siglas
        .3      symbols         Símbolos
        .4      chemical        Símbolos Químicos
    F.x     Chapter                 Capítulos
    G.x     Bibliography            Bibliografia
    H.x     Apendix                 Apêndices
    I.x     Annex                   Anexos
\end{verbatim}


\section{===== Glossário =====}

Para adiconar uma nova entrada de dicionário:

\begin{verbatim}
    \newglossaryentry{<comando>}{
        name={<Nome do Item>},
        sort={<Ordenar o Item>},
        description={<Descrição do Item>}
    }
\end{verbatim}


Para utilizar uma entrada de dicionário previamente adicionada basta usar:

\begin{verbatim}
    \gls{<comando>}
\end{verbatim}


\section{===== Acrónimos/Siglas =====}

Para adicionar uma nova sigla, basta ir ao ficheiro \verb|./Chapter/E2_acronyms|.tex e adicionar uma nova entrada segundo as seguintes orientações:

Acrónimo/Sigla Complexo:
\begin{verbatim}
    \newacronym[sort=nome]{comando}{SIGLA}{significado}
\end{verbatim}

Acrónimo/Sigla Simples:
\begin{verbatim}
    \newacronym{comando}{SIGLA}{significado}
\end{verbatim}

Acrónimo/Sigla Complexo com Plural:
\begin{verbatim}
    \newacronym[plural={SIGLAs}, longplural={significado_plural}]{comando}{SIGLA}{significado}
\end{verbatim}

Para utilizar uma sigla previamente adicionada basta usar:

\begin{verbatim}
    \gls{<comando>}
\end{verbatim}

ou desta forma caso se queira que a descrição da sigla comece com letra maiúscula:

\begin{verbatim}
    \Gls{<comando>}
\end{verbatim}

ou desta forma caso se queira utilizar a sigla no plural:

\begin{verbatim}
    \glspl{<comando>}
    \Glspl{<comando>}
\end{verbatim}

ou para utlizar a forma abrevida da sigla forçadamente:

\begin{verbatim}
    \acrshort{<comando>}
    \acrshortpl{<comando>}
\end{verbatim}

ou para utilizar a forma expandida da sigla forçadamente:

\begin{verbatim}
    \acrfull{<comando>}
\end{verbatim}


\bgroup
\rowcolors{1}{}{GhostWhite}
\begin{xltabular}{\textwidth}{Xccc}
    \caption{Resumo dos comandos do \texttt{glossaries} e \texttt{glossaries-extra}}
    \label{tab:glossaries_commands}\\
    \toprule
    \textbf{Função} & \textbf{Comando Antigo (\texttt{glossaries})} & \textbf{Novo Comando (\texttt{glossaries-extra})} & \textbf{Versão Capitalizada (\texttt{glossaries-extra})} \\
    \midrule
    Forma curta (singular)  & \texttt{\textbackslash acrshort\{\}}    & \texttt{\textbackslash glsxtrshort\{\}}    & \texttt{\textbackslash Glsxtrshort\{\}} \\
    Forma curta (plural)    & \texttt{\textbackslash acrshortpl\{\}}  & \texttt{\textbackslash glsxtrshortpl\{\}}  & \texttt{\textbackslash Glsxtrshortpl\{\}} \\
    Forma longa (singular)  & \texttt{\textbackslash acrlong\{\}}     & \texttt{\textbackslash glsxtrlong\{\}}     & \texttt{\textbackslash Glsxtrlong\{\}} \\
    Forma longa (plural)    & \texttt{\textbackslash acrlongpl\{\}}   & \texttt{\textbackslash glsxtrlongpl\{\}}   & \texttt{\textbackslash Glsxtrlongpl\{\}} \\
    Forma completa (singular) & \texttt{\textbackslash gls\{\}}       & \texttt{\textbackslash glsxtrfull\{\}}     & \texttt{\textbackslash Glsxtrfull\{\}} \\
    Forma completa (plural)   & \texttt{\textbackslash glspl\{\}}     & \texttt{\textbackslash glsxtrfullpl\{\}}   & \texttt{\textbackslash Glsxtrfullpl\{\}} \\
    \bottomrule
\end{xltabular}
\egroup


\section{===== Símbolos =====}

Para adicionar um novo símbolo:

\begin{verbatim}
    \glsxtrnewsymbol[sort=OrdenarSímbolo,description={Descrição do Símbolo}]{comando}{\ensuremath{Nome do Símbolo}}
\end{verbatim}

Para utilizar um símbolo previamente adicionada basta usar:

\begin{verbatim}
    \gls{<comando>}
\end{verbatim}


\section{===== Símbolos Químicos =====}

Para adicionar um novo símbolo químico:

\begin{verbatim}
    \newglossaryentry{chem:comando}% label
    {%
        type=chemical,% glossary type
        name={$Nome do Símbolo$},%
        description={Descrição do Símbolo},%
        sort={OrdenarSímbolo}%
    }
\end{verbatim}

Para utilizar um símbolo químico previamente adicionado basta usar:

\begin{verbatim}
    \gls{<chem:comando>}
\end{verbatim}


\section{===== Citações =====}

Na pasta \verb|./Bibliography| incluir todos os ficheiros .bib com estruturas de citações semelhantes a esta:

\begin{verbatim}
    @<tipo>{<comando>,
        author = {<autor>},
        title = {<título>},
        year = {<ano>},
        ...
    }
\end{verbatim}

Ao longo do texto, se for necessário uma citação de um único item devendo ser utilizado da seguinte forma:

\begin{verbatim}
    Como dizem os autores em~\cite{<comando>}
\end{verbatim}

ou de vários items de uma só vez:

\begin{verbatim}
    Como dizem os autores em~\cite{<comando1>,<comando2>,...}
\end{verbatim}


\section{===== Notas de Rodapé =====}

Para inserir uma nota de rodapé ao longo do texto, basta fazer da seguinte forma:

\begin{verbatim}
    Isto é um exemplo de nota de rodapé\footnote{Esta é a descrição da nota de rodapé.}
\end{verbatim}


\section{===== Tabelas =====}

Para criar tabelas deve ser utilizado o seguinte molde:

\begin{verbatim}
    \bgroup
    \rowcolors{1}{}{GhostWhite}
    \begin{xltabular}{\textwidth}{Xcc}
        \caption{Legenda da Tabela.}
        \label{tab:comando}\\
        \toprule
        \rowcolor{Gainsboro}%
        Coluna A        & Coluna B      & Coluna C \\
        \midrule
        Texto A1        & Texto B1      & Texto C1 \\
        Texto A2        & Texto B2      & Texto C2 \\
        Texto A3        & Texto B3      & Texto C3 \\
        \midrule
        \rowcolor{Gainsboro}%
        Total A         & Total B       & Total C \\
        \bottomrule
    \end{xltabular}
    \egroup
\end{verbatim}

\subsection{Exemplo}

\bgroup
\rowcolors{1}{}{GhostWhite}
\begin{xltabular}{\textwidth}{Xcc}
    \caption{Legenda da Tabela.}
    \label{tab:comando}\\
    \toprule
    \rowcolor{Gainsboro}%
    Coluna A        & Coluna B      & Coluna C \\
    \midrule
    Texto A1        & Texto B1      & Texto C1 \\
    Texto A2        & Texto B2      & Texto C2 \\
    Texto A3        & Texto B3      & Texto C3 \\
    \midrule
    \rowcolor{Gainsboro}%
    Total A         & Total B       & Total C \\
    \bottomrule
\end{xltabular}
\egroup


\section{===== Figuras =====}

Para colocar uma ou mais imagens agrupadas dentro da mesma figura, pode ser feito da seguinte forma:

\begin{verbatim}
    \begin{figure}[htbp]
        \centering
        \includegraphics[height=5cm]{nome_ficheiro_imagem} \\            % Altura fixa de 5 cm
        \includegraphics[width=10cm]{nome_ficheiro_imagem} \\            % Largura fixa de 10 cm
        \includegraphics[width=0.5\linewidth]{nome_ficheiro_imagem}   \\ % 50% da largura da linha
        \includegraphics[height=0.3\textheight]{nome_ficheiro_imagem} \\ % 30% da altura da linha
        \caption{Legenda da Imagem.}
        \label{fig:comando}
    \end{figure}
\end{verbatim}

Caso seja preciso criar subfiguras, pode ser feito da seguinte forma:

\begin{verbatim}
    \begin{figure}[htbp]
        \centering
        \subbottom[Legenda da Imagem 1.]{%
            \label{fig:subcomando1}
            \includegraphics[width=0.5\linewidth]{nome_ficheiro_imagem}
        }
        % espaçamento vertical ou horizontal, se necessário.
        \subbottom[Legenda da Imagem 2.]{%
            \label{fig:subcomando2}
            \includegraphics[width=0.5\linewidth]{nome_ficheiro_imagem}
        }
        \caption{Legenda da Imagem}
        \label{fig:comandoCompleto}
    \end{figure}
\end{verbatim}

\subsection{Exemplo}

\begin{figure}[htbp]
    \centering
    \includegraphics[height=2cm]{example-image-duck} \\            % Altura fixa de 5 cm
    \includegraphics[width=5cm]{example-image-duck} \\            % Largura fixa de 10 cm
    \includegraphics[width=0.5\linewidth]{example-image-duck}   \\ % 50% da largura da linha
    \includegraphics[height=0.3\textheight]{example-image-duck} \\ % 30% da altura da linha
    \caption{Legenda da Imagem.}
    \label{fig:comando}
\end{figure}

\begin{figure}[htbp]
    \centering
    \subbottom[Legenda da Imagem 1.]{%
        \label{fig:subcomando1}
        \includegraphics[width=0.5\linewidth]{example-image-duck}
    }
    % espaçamento vertical ou horizontal, se necessário.
    \subbottom[Legenda da Imagem 2.]{%
        \label{fig:subcomando2}
        \includegraphics[width=0.5\linewidth]{example-image-duck}
    }
    \caption{Legenda da Imagem}
    \label{fig:comandoCompleto}
\end{figure}


\section{===== Espaçamento Horizontal e Vertical =====}

Horizontal:
\begin{verbatim}
    \hspace{1cm}    % Espaçamento horizontal de 1cm
    \hfill          % Preenche o espaço horizontal disponível
\end{verbatim}

Vertical:
\begin{verbatim}
    \vspace{1cm}    % Espaçamento vertical de 1cm
    \vfill          % Preenche o espaço vertical disponível
\end{verbatim}

Relativo:
\begin{verbatim}
    \quad           % Aproximadamente 1 em.
    \qquad          % Aproximadamente 2 em.
\end{verbatim}


\section{===== Equações =====}

Para equações inline:

\begin{verbatim}
    $ax^2 + bx + c = 0$
\end{verbatim}

Para equações numa linha separada:

\begin{verbatim}
    \[ x = \frac{-b \pm \sqrt{b^2-4ac}}{2a} \]
\end{verbatim}

Para equações numeradas:

\begin{verbatim}
    \begin{equation}
        e = mc^2
        \label{eq:comando}
    \end{equation}
\end{verbatim}

que depois pode ser referenciado da seguinte forma:

\begin{verbatim}
    \ref{eq:comando}
\end{verbatim}


\section{===== Listas =====}

Para fazer listas, pode ser feito da seguinte forma:

\begin{verbatim}
    \begin{listing}[H]
        \begin{minted}{C}
            if(a==b)
                puts("YESS!")
        \end{minted}
        \caption{Legenda da Lista.}
        \label{lst:comando}
    \end{listing}
\end{verbatim}


\section{===== Estrutura de um Capítulo =====}

Um capítulo pode conter alguns destes items:

\begin{verbatim}
    \chapter{Nome do Capítulo}
    \label{cha:comando}

    \section{Nome da Seção}
    \label{sec:comando}

    \subsection{Nome da Subseção}
    \label{sub:comando}

    \subsubsection{Nome da Subsubseção}
    \label{ssub:comando}
\end{verbatim}


\section{===== Estrutura de um Apêndice =====}

Um apêndice pode conter alguns destes items:

\begin{verbatim}
    \chapter{Nome do Apêndice}
    \label{app:comando}

    \section{Nome da Seção}
    \label{sec:comando}

    \subsection{Nome da Subseção}
    \label{sub:comando}

    \subsubsection{Nome da Subsubseção}
    \label{ssub:comando}
\end{verbatim}


\section{===== Estrutura de um Anexo =====}

Um anexo pode conter alguns destes items:

\begin{verbatim}
    \chapter{Nome do Anexo}
    \label{ann:comando}

    \section{Nome da Seção}
    \label{sec:comando}

    \subsection{Nome da Subseção}
    \label{sub:comando}

    \subsubsection{Nome da Subsubseção}
    \label{ssub:comando}
\end{verbatim}


\section{===== Listas Enumeradas e com Marcas =====}

Para fazer listas enumeradas:

\begin{verbatim}
    \begin{enumerate}
        \item Texto 1.
        \item Texto 2.
        \item ...
    \end{enumerate}
\end{verbatim}

Para listas com marcas:

\begin{verbatim}
    \begin{itemize}
        \item Texto 1.
        \item Texto 2.
        \item ...
    \end{itemize}
\end{verbatim}


\section{===== Texto Verbatim (Literal) =====}

Para adiconar texto literal inline:

\begin{verbatim}
    \verb|texto|
\end{verbatim}

onde o caracter delimitador não deve estar presente no texto.

Para adicionar texto literal em linhas isoladas:

\begin{verbatim}
    \begin{verbatim}
        Texto.
    \ end{verbatim}
\end{verbatim}


\section{===== Página Horizontal =====}

Para criar uma página horizontal deve ser feito da seguinte forma:

\begin{verbatim}
    \begin{newpdflayout}{210mm}{297mm}%{420mm}

        Texto da Página Horizontal

    \end{newpdflayout}
\end{verbatim}
