%!TEX root = ../template.tex
%%%%%%%%%%%%%%%%%%%%%%%%%%%%%%%%%%%%%%%%%%%%%%%%%%%%%%%%%%%%%%%%%%%%
%% F1_introduction.tex
%% NOVA thesis document file
%%
%% Chapter with lots of dummy text
%%%%%%%%%%%%%%%%%%%%%%%%%%%%%%%%%%%%%%%%%%%%%%%%%%%%%%%%%%%%%%%%%%%%

\typeout{NT FILE F1_introduction.tex}%


%%%%%%%%%%%%%%%%%%%%%%%%%%%%%%%%%%%%%%%%%%%%%%%%%%
\chapter{Introdução}
\label{cha:introdução}
%%%%%%%%%%%%%%%%%%%%%%%%%%%%%%%%%%%%%%%%%%%%%%%%%%



%%%%%%%%%%%%%%%%%%%%%%%%%%%%%%%%%%%%%%%%%%%%%%%%%%
\section{Enquadramento e Motivação}
\label{sec:enquadramento_motivação}
%%%%%%%%%%%%%%%%%%%%%%%%%%%%%%%%%%%%%%%%%%%%%%%%%%

O progresso das tecnologias de comunicações e das redes veiculares ad hoc (VANETs) têm produzido avanços no desenvolvimento de sistemas de transporte intelegentes. Neste contexto a troca de dados de forma eficiente e fíavel entre veíclos representa um grande desafio.
O paradigma Named Data Network (NDN) aparece como uma alternativa às arquiteturas centradas em IP, 
pois priveligia uma comunicação com base em contéudos característica de ambientes muito dinâmicos como os das \gls{VANETs}.
Uma parte essencial do \gls{NDN} é a camada de sincronização, esta assegura a consistência e atualização dos dados, mas esses mecanismos de sincronização já existentes necessitam precisam de ser adaptadosao cenário das redes veiculares, principamente no que toca a latência, escalabilidade e à gestao de restrições geográficas e temporais.
Esta dissertação tem como objetivo desnevolver uma camada de sincronização robusta, especificamente desenhada para o \gls{NDN} aplicado às \gls{VANETs}, contribuindo para a fiabilidade e o melhor desempenho das comunicações veiculares em contextos de sistasms de transporte.



%%%%%%%%%%%%%%%%%%%%%%%%%%%%%%%%%%%%%%%%%%%%%%%%%%
\section{Objetivos}
\label{sec:objetivos}
%%%%%%%%%%%%%%%%%%%%%%%%%%%%%%%%%%%%%%%%%%%%%%%%%%
Os objetivos desta dissertação passam por:
\begin{itemize}
    \item Revisitar mecanismos já existentes ede sincronização no \gls{NDN}, neste caso o State Vector Sync \gls{SVSync} e identificar algumas limitações nas redes veiculares.
    \item Propor uma solução para a sincronização adaptada às \gls{VANETs}, incluindo as restrições geográficas e temporais
    \item Desenvolver  algoritmos para a sincronização de dados nas redes veiculares.
    \item Implementar a solução proposta
    \item Avaliar a eficácia da solução proposta.

\end{itemize}


%%%%%%%%%%%%%%%%%%%%%%%%%%%%%%%%%%%%%%%%%%%%%%%%%%
%\section{Organização do Documento}
%\label{sec:organização_documento}
%%%%%%%%%%%%%%%%%%%%%%%%%%%%%%%%%%%%%%%%%%%%%%%%%%

